\documentclass{article}

\newcommand{\version}{@LIBSUPERMESH_VERSION@}


\usepackage{amsmath}
\usepackage{amssymb}
\usepackage{enumitem}
\usepackage[a4paper,left=20mm,right=20mm]{geometry}
\usepackage[pdfborder={0 0 0}]{hyperref}
\usepackage{listings}
\usepackage{natbib}

\renewcommand{\ttdefault}{pcr}

\begin{document}

\begin{titlepage}
\begin{center}

\Huge{libsupermesh version \version} \\[0.15\textheight]
\Large{Dr. Patrick E. Farrell, Dr. J. R. Maddison, \\ Mr. I. Panourgias}

\end{center}
\end{titlepage}

\tableofcontents

\section{Introduction}

libsupermesh is a Fortran library for serial and parallel mesh intersection, or
supermeshing \citep{farrell2009,farrell2009a}. libsupermesh specifically
includes components used in the local supermeshing approach described in
\citet{farrell2011}. Here, given two input finite element meshes, elements in
each mesh which intersect are identified, and a local mesh of their intersection
is generated.

\section{Building libsupermesh}

libsupermesh is built using CMake\footnote{\url{https://cmake.org/}}, for
example via:
\begin{lstlisting}[language=sh,frame=single]
cd [LIBSUPERMESH_SOURCE_PATH]/build
cmake [LIBSUPERMESH_SOURCE_PATH] \
  -DCMAKE_INSTALL_PREFIX=[LIBSUPERMESH_INSTALL_PATH]
make
make install
\end{lstlisting}
where \verb+[LIBSUPERMESH_SOURCE_PATH]+ is the path to the source directory, and
\linebreak \verb+[LIBSUPERMESH_INSTALL_PATH]+ is the path to the installation
directory. Note that libsupermesh cannot be built in the source directory.

\subsection*{CMake variables}

\begin{description}[font=\ttfamily\bfseries]
  \item[CMAKE\_BUILD\_TYPE] Build type. Valid options are ``Release'' and
    ``Debug''. Default ``Release''.
  \item[BUILD\_SHARED\_LIBS] Build a shared library. Default disabled. 
  \item[LIBSUPERMESH\_AUTO\_COMPILER\_FLAGS] Choose compiler flags
    automatically, enabling aggressive optimisation options for the Release
    build and additional debugging options for the Debug build. Active only if
    the GNU or Cray compilers are used. This overrides other CMake compiler flag
    variables. Default enabled.
  \item[LIBSUPERMESH\_ENABLE\_JUDY] Use the Judy library. Requires an available
    installation of Judy. Default disabled.
  \item[LIBSUPERMESH\_ENABLE\_TIMERS] Enable internal timers for parallel
    supermeshing. Default disabled.
  \item[LIBSUPERMESH\_OVERLAP\_COMPUTE\_COMMS] Overlaps communication and some
    calculation in parallel supermeshing. Default disabled.
\end{description}

\subsection*{Build targets}

\begin{description}[font=\ttfamily\bfseries]
  \item[make] Default build. Builds the library and unit tests.
  \item[make test] Runs the unit tests.
  \item[make doc] Builds this manual.
\end{description}

\section{Using libsupermesh}

The \verb+libsupermesh+ Fortran module provides access to all functionality
described in this manual. This can be accessed via:

\begin{lstlisting}[language=FORTRAN]
  use libsupermesh
\end{lstlisting}

\section{Intersection identification}

libsupermesh includes a number of different algorithms for identifying candidate
pairs of elements, in two different meshes, which intersect. These algorithms
are all based upon an ``axis-aligned bounding box'' (AABB) intersection
predicate. Hence the different approaches are all guaranteed to return all pairs
of elements which have a non-trivial intersection\footnote{Except for numerical
precision related errors.}, but may additionally return pairs of elements which
have zero intersection.

The following definitions are used:
\begin{description}[leftmargin=\parindent,labelindent=\parindent]
  \item[$d$] Mesh A defines a subset $\Omega_A \subset \mathbb{R}^d$, and mesh B
    a subset $\Omega_B \subset \mathbb{R}^d$.
  \item[$l_A$] Number of nodes per element in mesh A.
  \item[$l_B$] Number of nodes per element in mesh B.
  \item[$V_A$] Number of nodes in mesh A.
  \item[$V_B$] Number of nodes in mesh B.
  \item[$E_A$] Number of elements in mesh A.
  \item[$E_B$] Number of elements in mesh B.
\end{description}

Element-node graphs are represented as $l_A \times E_A$ and
$l_B \times E_B$ integer arrays respectively. For example, if \verb+enlist_a+
stores the element-node graph for mesh A, then in the following:
\begin{lstlisting}[language=FORTRAN]
  integer, dimension(l_A, E_A) :: enlist_a
  integer, dimension(l_A) :: nodes_a
  
  ! ...

  nodes_a = enlist_a(:, ele_a)
\end{lstlisting}
\verb+nodes_a+ contains the indices of nodes in mesh A for element \verb+ele_a+.

Candidate intersection element-element graphs are stored either as a ragged
array or as a Compressed Sparse Row (CSR) sparsity pattern\footnote{See the
\href{http://docs.scipy.org/doc/scipy-0.15.1/reference/generated/scipy.sparse.csr_matrix.html}{SciPy documentation.}}.
For the ragged array variant:
\begin{lstlisting}[language=FORTRAN]
  type(intersections), dimension(E_A) :: map_ab
  integer :: n
  integer, dimension(:), pointer :: v
  
  ! ...

  n = map_ab(ele_a)%n
  v => map_ab(ele_a)%v
\end{lstlisting}
or for the CSR sparsity pattern variant:
\begin{lstlisting}[language=FORTRAN]
  integer, dimension(:), allocatable, target :: map_ab_indices
  integer, dimension(E_A + 1) :: map_ab_indptr
  integer :: n
  integer, dimension(:), pointer :: v
  
  ! ...

  n = map_ab_indptr(ele_a + 1) - map_ab_indptr(ele_a)
  v => map_ab_indices(map_ab_indptr(ele_a):map_ab_indptr(ele_a + 1) - 1)
\end{lstlisting}
\verb+n+ contains the number of candidate elements in mesh B which may intersect
with element \verb+ele_a+ of mesh A. \verb+v+ points at a vector of length
\verb+n+ containing the indices of elements in mesh B which may intersect with
element \verb+ele_a+ of mesh A.

\subsection{Identifying all candidate intersections}

The following intersection finders return an element-element graph which
associates with all elements in a mesh ``A'' candidate intersecting elements in
the mesh ``B''. 

\subsubsection{\texttt{sort\_intersection\_finder}}

An intersection finder for 1D meshes, based upon a combination of a merge sort
and a 1D advancing front.

\begin{lstlisting}[language=FORTRAN]
  interface sort_intersection_finder
    module procedure sort_intersection_finder_rank_1_intersections, &
      & sort_intersection_finder_rank_2_intersections, &
      & sort_intersection_finder_rank_1_csr_sparsity, &
      & sort_intersection_finder_rank_2_csr_sparsity
  end interface sort_intersection_finder
\end{lstlisting}

\begin{lstlisting}[language=FORTRAN]
  pure subroutine sort_intersection_finder_rank_1_intersections( &
    & positions_a, enlist_a, positions_b, enlist_b, map_ab)
    real, dimension(:), intent(in) :: positions_a
    integer, dimension(:, :), intent(in) :: enlist_a
    real, dimension(:), intent(in) :: positions_b
    integer, dimension(:, :), intent(in) :: enlist_b
    type(intersections), dimension(:), intent(out) :: map_ab
\end{lstlisting}

\begin{description}[font=\ttfamily\bfseries,leftmargin=2.2\parindent,labelindent=1.7\parindent,noitemsep]
  \item[positions\_a] Length $V_A$ vector. Mesh A node coordinates.
  \item[enlist\_a] $2 \times E_A$ array. Mesh A element-node graph.
  \item[positions\_b] Length $V_B$ vector. Mesh B node coordinates.
  \item[enlist\_b] $2 \times E_B$ array. Mesh B element-node graph.
  \item[map\_ab] Length $E_A$ vector. Candidate intersection element-element
    graph.
\end{description}

\begin{lstlisting}[language=FORTRAN]
  pure subroutine sort_intersection_finder_rank_2_intersections( &
    & positions_a, enlist_a, positions_b, enlist_b, map_ab)
    real, dimension(:, :), intent(in) :: positions_a
    integer, dimension(:, :), intent(in) :: enlist_a
    real, dimension(:, :), intent(in) :: positions_b
    integer, dimension(:, :), intent(in) :: enlist_b
    type(intersections), dimension(:), intent(out) :: map_ab
\end{lstlisting}

\begin{description}[font=\ttfamily\bfseries,leftmargin=2.2\parindent,labelindent=1.7\parindent,noitemsep]
  \item[positions\_a] $1 \times V_A$ array. Mesh A node coordinates.
  \item[enlist\_a] $2 \times E_A$ array. Mesh A element-node graph.
  \item[positions\_b] $1 \times V_B$ array. Mesh B node coordinates.
  \item[enlist\_b] $2 \times E_B$ array. Mesh B element-node graph.
  \item[map\_ab] Length $E_A$ vector. Candidate intersection element-element
    graph.
\end{description}

\begin{lstlisting}[language=FORTRAN]
  pure subroutine sort_intersection_finder_rank_1_csr_sparsity( &
    & positions_a, enlist_a, positions_b, enlist_b, &
    & map_ab_indices, map_ab_indptr)
    real, dimension(:), intent(in) :: positions_a
    integer, dimension(:, :), intent(in) :: enlist_a
    real, dimension(:), intent(in) :: positions_b
    integer, dimension(:, :), intent(in) :: enlist_b
    integer, dimension(:), allocatable, intent(out) :: map_ab_indices
    integer, dimension(:), intent(out) :: map_ab_indptr
\end{lstlisting}

\begin{description}[font=\ttfamily\bfseries,leftmargin=2.2\parindent,labelindent=1.7\parindent,noitemsep]
  \item[positions\_a] Length $V_A$ vector. Mesh A node coordinates.
  \item[enlist\_a] $2 \times E_A$ array. Mesh A element-node graph.
  \item[positions\_b] Length $V_B$ vector. Mesh B node coordinates.
  \item[enlist\_b] $2 \times E_B$ array. Mesh B element-node graph.
  \item[map\_ab\_indices, map\_ab\_indptr] CSR sparsity. Candidate intersection
    element-element graph. \linebreak \verb+map_ab_indices+ has length
    $E_A + 1$.
\end{description}

\begin{lstlisting}[language=FORTRAN]
  pure subroutine sort_intersection_finder_rank_2_csr_sparsity( &
    & positions_a, enlist_a, positions_b, enlist_b, &
    & map_ab_indices, map_ab_indptr)
    real, dimension(:, :), intent(in) :: positions_a
    integer, dimension(:, :), intent(in) :: enlist_a
    real, dimension(:, :), intent(in) :: positions_b
    integer, dimension(:, :), intent(in) :: enlist_b
    integer, dimension(:), allocatable, intent(out) :: map_ab_indices
    integer, dimension(:), intent(out) :: map_ab_indptr
\end{lstlisting}

\begin{description}[font=\ttfamily\bfseries,leftmargin=2.2\parindent,labelindent=1.7\parindent,noitemsep]
  \item[positions\_a] $1 \times V_A$ array. Mesh A node coordinates.
  \item[enlist\_a] $2 \times E_A$ array. Mesh A element-node graph.
  \item[positions\_b] $1 \times V_B$ array. Mesh B node coordinates.
  \item[enlist\_b] $2 \times E_B$ array. Mesh B element-node graph.
  \item[map\_ab\_indices, map\_ab\_indptr] CSR sparsity. Candidate intersection
    element-element graph. \linebreak \verb+map_ab_indices+ has length 
    $E_A + 1$.
\end{description}

\subsubsection{\texttt{quadtree\_intersection\_finder}}\label{sect:quadtree_global}

An intersection finder for 2D meshes, based upon a quadtree data structure. Each
interface accepts an optional \verb+max_size+ argument, which specifies the
maximum number of elements in a leaf node. This should be greater than the
maximum number of elements which can be found in a neighbourhood of an arbitrary
point. \verb+max_size+ defaults to the maximum of $256$, and the maximal
node-element graph degree, if not specified.

\begin{lstlisting}[language=FORTRAN]  
  pure function max_nelist_degree(nnodes, enlist) result(max_degree)
    integer, intent(in) :: nnodes
    integer, dimension(:, :), intent(in) :: enlist
    integer :: max_degree
\end{lstlisting}

\begin{description}[font=\ttfamily\bfseries,leftmargin=2.2\parindent,labelindent=1.7\parindent,noitemsep]
  \item[nnodes] $V_B$.
  \item[enlist] $l_B \times E_B$ array. Mesh B element-node graph.
  \item[max\_degree] Maximal mesh B node-element graph degree.
\end{description}

\begin{lstlisting}[language=FORTRAN]  
  interface quadtree_intersection_finder
    module procedure quadtree_intersection_finder_intersections, &
      & quadtree_intersection_finder_csr_sparsity
  end interface quadtree_intersection_finder
\end{lstlisting}

\begin{lstlisting}[language=FORTRAN]
  subroutine quadtree_intersection_finder_intersections( &
    & positions_a, enlist_a, positions_b, enlist_b, map_ab, max_size)
    real, dimension(:, :), intent(in) :: positions_a
    integer, dimension(:, :), intent(in) :: enlist_a
    real, dimension(:, :), intent(in) :: positions_b
    integer, dimension(:, :), intent(in) :: enlist_b
    type(intersections), dimension(:), intent(out) :: map_ab
    integer, optional, intent(in) :: max_size
\end{lstlisting}

\begin{description}[font=\ttfamily\bfseries,leftmargin=2.2\parindent,labelindent=1.7\parindent,noitemsep]
  \item[positions\_a] $d \times V_A$ array. Mesh A node coordinates.
  \item[enlist\_a] $l_A \times E_A$ array. Mesh A element-node graph.
  \item[positions\_b] $d \times V_B$ array. Mesh B node coordinates.
  \item[enlist\_b] $l_B \times E_B$ array. Mesh B element-node graph.
  \item[map\_ab] Length $E_A$ vector. Candidate intersection element-element
    graph.
  \item[max\_size] Maximum number of elements in a quadtree leaf node.
\end{description}
  
\begin{lstlisting}[language=FORTRAN]
  subroutine quadtree_intersection_finder_csr_sparsity( &
    & positions_a, enlist_a, positions_b, enlist_b, &
    & map_ab_indices, map_ab_indptr, max_size)
    real, dimension(:, :), intent(in) :: positions_a
    integer, dimension(:, :), intent(in) :: enlist_a
    real, dimension(:, :), intent(in) :: positions_b
    integer, dimension(:, :), intent(in) :: enlist_b
    integer, dimension(:), allocatable, intent(out) :: map_ab_indices
    integer, dimension(:), intent(out) :: map_ab_indptr
    integer, optional, intent(in) :: max_size
\end{lstlisting}

\begin{description}[font=\ttfamily\bfseries,leftmargin=2.2\parindent,labelindent=1.7\parindent,noitemsep]
  \item[positions\_a] $d \times V_A$ array. Mesh A node coordinates.
  \item[enlist\_a] $l_A \times E_A$ array. Mesh A element-node graph.
  \item[positions\_b] $d \times V_B$ array. Mesh B node coordinates.
  \item[enlist\_b] $l_B \times E_B$ array. Mesh B element-node graph.
  \item[map\_ab\_indices, map\_ab\_indptr] CSR sparsity. Candidate intersection
    element-element graph. \linebreak \verb+map_ab_indices+ has length
    $E_A + 1$.
  \item[max\_size] Maximum number of elements in a quadtree leaf node.
\end{description}

\subsubsection{\texttt{octtree\_intersection\_finder}}\label{sect:octtree_global}

An intersection finder for 3D meshes, based upon an octtree data structure. See
section \ref{sect:quadtree_global} for details regarding the optional
\verb+max_size+ argument.

\begin{lstlisting}[language=FORTRAN]  
  interface octtree_intersection_finder
    module procedure octtree_intersection_finder_intersections, &
      & octtree_intersection_finder_csr_sparsity
  end interface octtree_intersection_finder
\end{lstlisting}

\begin{lstlisting}[language=FORTRAN]
  subroutine octtree_intersection_finder_intersections( &
    & positions_a, enlist_a, positions_b, enlist_b, map_ab, max_size)
    real, dimension(:, :), intent(in) :: positions_a
    integer, dimension(:, :), intent(in) :: enlist_a
    real, dimension(:, :), intent(in) :: positions_b
    integer, dimension(:, :), intent(in) :: enlist_b
    type(intersections), dimension(:), intent(out) :: map_ab
    integer, optional, intent(in) :: max_size
\end{lstlisting}

\begin{description}[font=\ttfamily\bfseries,leftmargin=2.2\parindent,labelindent=1.7\parindent,noitemsep]
  \item[positions\_a] $d \times V_A$ array. Mesh A node coordinates.
  \item[enlist\_a] $l_A \times E_A$ array. Mesh A element-node graph.
  \item[positions\_b] $d \times V_B$ array. Mesh B node coordinates.
  \item[enlist\_b] $l_B \times E_B$ array. Mesh B element-node graph.
  \item[map\_ab] Length $E_A$ vector. Candidate intersection element-element
    graph.
  \item[max\_size] Maximum number of elements in an octtree leaf node.
\end{description}
  
\begin{lstlisting}[language=FORTRAN]
  subroutine octtree_intersection_finder_csr_sparsity( &
    & positions_a, enlist_a, positions_b, enlist_b, &
    & map_ab_indices, map_ab_indptr, max_size)
    real, dimension(:, :), intent(in) :: positions_a
    integer, dimension(:, :), intent(in) :: enlist_a
    real, dimension(:, :), intent(in) :: positions_b
    integer, dimension(:, :), intent(in) :: enlist_b
    integer, dimension(:), allocatable, intent(out) :: map_ab_indices
    integer, dimension(:), intent(out) :: map_ab_indptr
    integer, optional, intent(in) :: max_size
\end{lstlisting}

\begin{description}[font=\ttfamily\bfseries,leftmargin=2.2\parindent,labelindent=1.7\parindent,noitemsep]
  \item[positions\_a] $d \times V_A$ array. Mesh A node coordinates.
  \item[enlist\_a] $l_A \times E_A$ array. Mesh A element-node graph.
  \item[positions\_b] $d \times V_B$ array. Mesh B node coordinates.
  \item[enlist\_b] $l_B \times E_B$ array. Mesh B element-node graph.
  \item[map\_ab\_indices, map\_ab\_indptr] CSR sparsity. Candidate intersection
    element-element graph. \linebreak \verb+map_ab_indices+ has length
    $E_A + 1$.
  \item[max\_size] Maximum number of elements in an octtree leaf node.
\end{description}

\subsubsection{\texttt{tree\_intersection\_finder}}

An intersection finder for 2D and 3D meshes, which uses either the
\verb+quadtree_intersection_finder+ (section \ref{sect:quadtree_global}) or
\verb+octtree_intersection_finder+ (section \ref{sect:octtree_global}).

\begin{lstlisting}[language=FORTRAN]
  interface tree_intersection_finder
    module procedure tree_intersection_finder_intersections, &
      & tree_intersection_finder_csr_sparsity
  end interface tree_intersection_finder
\end{lstlisting}

\begin{lstlisting}[language=FORTRAN]
  subroutine tree_intersection_finder_intersections( &
    & positions_a, enlist_a, positions_b, enlist_b, map_ab)
    real, dimension(:, :), intent(in) :: positions_a
    integer, dimension(:, :), intent(in) :: enlist_a
    real, dimension(:, :), intent(in) :: positions_b
    integer, dimension(:, :), intent(in) :: enlist_b
    type(intersections), dimension(:), intent(out) :: map_ab
\end{lstlisting}

\begin{description}[font=\ttfamily\bfseries,leftmargin=2.2\parindent,labelindent=1.7\parindent,noitemsep]
  \item[positions\_a] $d \times V_A$ array. Mesh A node coordinates.
  \item[enlist\_a] $l_A \times E_A$ array. Mesh A element-node graph.
  \item[positions\_b] $d \times V_B$ array. Mesh B node coordinates.
  \item[enlist\_b] $l_B \times E_B$ array. Mesh B element-node graph.
  \item[map\_ab] Length $E_A$ vector. Candidate intersection element-element
    graph.
\end{description}

\begin{lstlisting}[language=FORTRAN]
  subroutine tree_intersection_finder_csr_sparsity( &
    & positions_a, enlist_a, positions_b, enlist_b, &
    & map_ab_indices, map_ab_indptr)
    real, dimension(:, :), intent(in) :: positions_a
    integer, dimension(:, :), intent(in) :: enlist_a
    real, dimension(:, :), intent(in) :: positions_b
    integer, dimension(:, :), intent(in) :: enlist_b
    integer, dimension(:), allocatable, intent(out) :: map_ab_indices
    integer, dimension(:), intent(out) :: map_ab_indptr
\end{lstlisting}

\begin{description}[font=\ttfamily\bfseries,leftmargin=2.2\parindent,labelindent=1.7\parindent,noitemsep]
  \item[positions\_a] $d \times V_A$ array. Mesh A node coordinates.
  \item[enlist\_a] $l_A \times E_A$ array. Mesh A element-node graph.
  \item[positions\_b] $d \times V_B$ array. Mesh B node coordinates.
  \item[enlist\_b] $l_B \times E_B$ array. Mesh B element-node graph.
  \item[map\_ab\_indices, map\_ab\_indptr] CSR sparsity. Candidate intersection
    element-element graph. \linebreak \verb+map_ab_indices+ has length
    $E_A + 1$.
\end{description}

\subsubsection{\texttt{rtree\_intersection\_finder}}

An intersection finder for 2D and 3D meshes, using a libspatialindex 1.8.5
R${}^*$-tree\footnote{\url{https://libspatialindex.github.io/}}.

\begin{lstlisting}[language=FORTRAN]
  interface rtree_intersection_finder
    module procedure rtree_intersection_finder_intersections, &
      & rtree_intersection_finder_csr_sparsity
  end interface rtree_intersection_finder
\end{lstlisting}

\begin{lstlisting}[language=FORTRAN]
  subroutine rtree_intersection_finder_intersections( &
    & positions_a, enlist_a, positions_b, enlist_b, map_ab)
    real(kind = c_double), dimension(:, :), intent(in) :: positions_a
    integer(kind = c_int), dimension(:, :), intent(in) :: enlist_a
    real(kind = c_double), dimension(:, :), intent(in) :: positions_b
    integer(kind = c_int), dimension(:, :), intent(in) :: enlist_b
    type(intersections), dimension(:), intent(out) :: map_ab
\end{lstlisting}

\begin{description}[font=\ttfamily\bfseries,leftmargin=2.2\parindent,labelindent=1.7\parindent,noitemsep]
  \item[positions\_a] $d \times V_A$ array. Mesh A node coordinates.
  \item[enlist\_a] $l_A \times E_A$ array. Mesh A element-node graph.
  \item[positions\_b] $d \times V_B$ array. Mesh B node coordinates.
  \item[enlist\_b] $l_B \times E_B$ array. Mesh B element-node graph.
  \item[map\_ab] Length $E_A$ vector. Candidate intersection element-element
    graph.
\end{description}

\begin{lstlisting}[language=FORTRAN]
  subroutine rtree_intersection_finder_csr_sparsity( &
    & positions_a, enlist_a, positions_b, enlist_b, &
    & map_ab_indices, map_ab_indptr)
    real, dimension(:, :), intent(in) :: positions_a
    integer, dimension(:, :), intent(in) :: enlist_a
    real, dimension(:, :), intent(in) :: positions_b
    integer, dimension(:, :), intent(in) :: enlist_b
    integer, dimension(:), allocatable, intent(out) :: map_ab_indices
    integer, dimension(:), intent(out) :: map_ab_indptr
\end{lstlisting}

\begin{description}[font=\ttfamily\bfseries,leftmargin=2.2\parindent,labelindent=1.7\parindent,noitemsep]
  \item[positions\_a] $d \times V_A$ array. Mesh A node coordinates.
  \item[enlist\_a] $l_A \times E_A$ array. Mesh A element-node graph.
  \item[positions\_b] $d \times V_B$ array. Mesh B node coordinates.
  \item[enlist\_b] $l_B \times E_B$ array. Mesh B element-node graph.
  \item[map\_ab\_indices, map\_ab\_indptr] CSR sparsity. Candidate intersection
    element-element graph. \linebreak \verb+map_ab_indices+ has length
    $E_A + 1$.
\end{description}

\subsubsection{\texttt{advancing\_front\_intersection\_finder}}

An intersection finder for meshes with simplex or hypercubic elements, using
the algorithm described in \citet{farrell2011}
\citep[see also][]{gander2009,gander2013}. The intersection finder is primarily
intended for the case where the meshes A and B cover the same domain. However
correct output is guaranteed if:
\begin{enumerate}
  \item Element $1$ of mesh A intersects with at least one element of mesh B.
  \item The elements of mesh B which intersect with any given element of mesh A
        define an element-element sub-graph of the mesh B element-element
        graph\footnote{Here element-element graphs are defined by elements
        sharing a common facet.}. All of these sub-graphs should be connected. 
\end{enumerate}

\begin{lstlisting}[language=FORTRAN]
  interface advancing_front_intersection_finder
    module procedure advancing_front_intersection_finder_intersections, &
      & advancing_front_intersection_finder_csr_sparsity
  end interface advancing_front_intersection_finder
\end{lstlisting}

\begin{lstlisting}[language=FORTRAN]
  subroutine advancing_front_intersection_finder_intersections( &
    & positions_a, enlist_a, positions_b, enlist_b, &
    & map_ab)
    real, dimension(:, :), intent(in) :: positions_a
    integer, dimension(:, :), intent(in) :: enlist_a
    real, dimension(:, :), intent(in) :: positions_b
    integer, dimension(:, :), intent(in) :: enlist_b
    type(intersections), dimension(:), intent(out) :: map_ab
\end{lstlisting}

\begin{description}[font=\ttfamily\bfseries,leftmargin=2.2\parindent,labelindent=1.7\parindent,noitemsep]
  \item[positions\_a] $d \times V_A$ array. Mesh A node coordinates.
  \item[enlist\_a] $l_A \times E_A$ array. Mesh A element-node graph.
  \item[positions\_b] $d \times V_B$ array. Mesh B node coordinates.
  \item[enlist\_b] $l_B \times E_B$ array. Mesh B element-node graph.
  \item[map\_ab] Length $E_A$ vector. Candidate intersection element-element
    graph.
\end{description}

\begin{lstlisting}[language=FORTRAN]
  subroutine advancing_front_intersection_finder_csr_sparsity( &
    & positions_a, enlist_a, positions_b, enlist_b, &
    & map_ab_indices, map_ab_indptr)
    real, dimension(:, :), intent(in) :: positions_a
    integer, dimension(:, :), intent(in) :: enlist_a
    real, dimension(:, :), intent(in) :: positions_b
    integer, dimension(:, :), intent(in) :: enlist_b
    integer, dimension(:), allocatable, intent(out) :: map_ab_indices
    integer, dimension(:), intent(out) :: map_ab_indptr
\end{lstlisting}

\begin{description}[font=\ttfamily\bfseries,leftmargin=2.2\parindent,labelindent=1.7\parindent,noitemsep]
  \item[positions\_a] $d \times V_A$ array. Mesh A node coordinates.
  \item[enlist\_a] $l_A \times E_A$ array. Mesh A element-node graph.
  \item[positions\_b] $d \times V_B$ array. Mesh B node coordinates.
  \item[enlist\_b] $l_B \times E_B$ array. Mesh B element-node graph.
  \item[map\_ab\_indices, map\_ab\_indptr] CSR sparsity. Candidate intersection
    element-element graph. \linebreak \verb+map_ab_indices+ has length
    $E_A + 1$.
\end{description}

\subsubsection{\texttt{brute\_force\_intersection\_finder}}

A basic brute force intersection finder with quadratic complexity.

\begin{lstlisting}[language=FORTRAN]
  interface brute_force_intersection_finder
    module procedure brute_force_intersection_finder_intersections, &
      & brute_force_intersection_finder_csr_sparsity
  end interface brute_force_intersection_finder
\end{lstlisting}

\begin{lstlisting}[language=FORTRAN]
  pure subroutine brute_force_intersection_finder_intersections( &
    & positions_a, enlist_a, positions_b, enlist_b, map_ab)
    real, dimension(:, :), intent(in) :: positions_a
    integer, dimension(:, :), intent(in) :: enlist_a
    real, dimension(:, :), intent(in) :: positions_b
    integer, dimension(:, :), intent(in) :: enlist_b
    type(intersections), dimension(:), intent(out) :: map_ab
\end{lstlisting}

\begin{description}[font=\ttfamily\bfseries,leftmargin=2.2\parindent,labelindent=1.7\parindent,noitemsep]
  \item[positions\_a] $d \times V_A$ array. Mesh A node coordinates.
  \item[enlist\_a] $l_A \times E_A$ array. Mesh A element-node graph.
  \item[positions\_b] $d \times V_B$ array. Mesh B node coordinates.
  \item[enlist\_b] $l_B \times E_B$ array. Mesh B element-node graph.
  \item[map\_ab] Length $E_A$ vector. Candidate intersection element-element
    graph.
\end{description}

\begin{lstlisting}[language=FORTRAN]
  pure subroutine brute_force_intersection_finder_csr_sparsity( &
    & positions_a, enlist_a, positions_b, enlist_b, &
    & map_ab_indices, map_ab_indptr)
    real, dimension(:, :), intent(in) :: positions_a
    integer, dimension(:, :), intent(in) :: enlist_a
    real, dimension(:, :), intent(in) :: positions_b
    integer, dimension(:, :), intent(in) :: enlist_b
    integer, dimension(:), allocatable, intent(out) :: map_ab_indices
    integer, dimension(:), intent(out) :: map_ab_indptr
\end{lstlisting}

\begin{description}[font=\ttfamily\bfseries,leftmargin=2.2\parindent,labelindent=1.7\parindent,noitemsep]
  \item[positions\_a] $d \times V_A$ array. Mesh A node coordinates.
  \item[enlist\_a] $l_A \times E_A$ array. Mesh A element-node graph.
  \item[positions\_b] $d \times V_B$ array. Mesh B node coordinates.
  \item[enlist\_b] $l_B \times E_B$ array. Mesh B element-node graph.
  \item[map\_ab\_indices, map\_ab\_indptr] CSR sparsity. Candidate intersection
    element-element graph. \linebreak \verb+map_ab_indices+ has length
    $E_A + 1$.
\end{description}

\subsection{Identifying candidate intersections using queryable trees}

The following intersection finders use a given mesh B to construct a tree based
structure which can be queried, one mesh A element at a time, for intersecting
elements.

\subsubsection{\texttt{quadtree\_type}}\label{sect:quadtree_query}
  
An intersection finder for 2D meshes, based upon a quadtree data structure. See
section \ref{sect:quadtree_global} for details regarding the optional
\verb+max_size+ argument.

~\newline
Allocate a \verb+quadtree_type+, which can later be queried for intersecting
elements.

\begin{lstlisting}[language=FORTRAN]
  interface allocate
    module procedure allocate_quadtree
  end interface allocate
\end{lstlisting}
  
\begin{lstlisting}[language=FORTRAN]
  subroutine allocate_quadtree(quadtree, positions, enlist, max_size)
    type(quadtree_type), intent(out) :: quadtree
    real, dimension(:, :), intent(in) :: positions
    integer, dimension(:, :), intent(in) :: enlist
    integer, optional, intent(in) :: max_size 
\end{lstlisting}

\begin{description}[font=\ttfamily\bfseries,leftmargin=2.2\parindent,labelindent=1.7\parindent,noitemsep]
  \item[quadtree] Quadtree data structure.
  \item[positions] $d \times V_B$ array. Mesh B node coordinates.
  \item[enlist] $l_A \times E_B$ array. Mesh B element-node graph.
  \item[max\_size] Maximum number of elements in a quadtree leaf node.
\end{description}

\noindent Query an allocated \verb+quadtree_type+ for intersecting elements.
  
\begin{lstlisting}[language=FORTRAN]
  interface query
    module procedure query_quadtree_allocatable, query_quadtree_pointer
  end interface query
\end{lstlisting}
    
\begin{lstlisting}[language=FORTRAN]
  subroutine query_quadtree_allocatable(quadtree, element_a, eles_b)
    type(quadtree_type), intent(inout) :: quadtree
    real, dimension(:, :), intent(in) :: element_a
    integer, dimension(:), allocatable, intent(out) :: eles_b
\end{lstlisting}
  
\begin{lstlisting}[language=FORTRAN]
  subroutine query_quadtree_pointer(quadtree, element_a, eles_b)
    type(quadtree_type), intent(inout) :: quadtree
    real, dimension(:, :), intent(in) :: element_a
    integer, dimension(:), pointer, intent(out) :: eles_b
\end{lstlisting}

\begin{description}[font=\ttfamily\bfseries,leftmargin=2.2\parindent,labelindent=1.7\parindent,noitemsep]
  \item[quadtree] Quadtree data structure.
  \item[element\_a] $d \times l_A$ array. Node coordinates of a mesh A element.
  \item[eles\_b] Elements of mesh B whose AABBs intersect with the AABB of the
    given mesh A element.
\end{description}

\noindent Deallocate a \verb+quadtree_type+.
  
\begin{lstlisting}[language=FORTRAN]
  interface deallocate
    module procedure deallocate_quadtree
  end interface deallocate
\end{lstlisting}
  
\begin{lstlisting}[language=FORTRAN]
  subroutine deallocate_quadtree(quadtree)
    type(quadtree_type), intent(inout) :: quadtree
\end{lstlisting}

\begin{description}[font=\ttfamily\bfseries,leftmargin=2.2\parindent,labelindent=1.7\parindent,noitemsep]
  \item[quadtree] Quadtree data structure.
\end{description}

\subsubsection{\texttt{octtree\_type}}\label{sect:octtree_query}
  
An intersection finder for 3D meshes, based upon an octtree data structure. See
section \ref{sect:octtree_global} for details regarding the optional
\verb+max_size+ argument.

~\newline
Allocate an \verb+octtree_type+, which can later be queried for intersecting
elements.

\begin{lstlisting}[language=FORTRAN]
  interface allocate
    module procedure allocate_octtree
  end interface allocate
\end{lstlisting}
  
\begin{lstlisting}[language=FORTRAN]
  subroutine allocate_octtree(octtree, positions, enlist, max_size)
    type(octtree_type), intent(out) :: octtree
    real, dimension(:, :), intent(in) :: positions
    integer, dimension(:, :), intent(in) :: enlist
    integer, optional, intent(in) :: max_size 
\end{lstlisting}

\begin{description}[font=\ttfamily\bfseries,leftmargin=2.2\parindent,labelindent=1.7\parindent,noitemsep]
  \item[octtree] Octtree data structure.
  \item[positions] $d \times V_B$ array. Mesh B node coordinates.
  \item[enlist] $l_A \times E_B$ array. Mesh B element-node graph.
  \item[max\_size] Maximum number of elements in an octtree leaf node.
\end{description}

\noindent Query an allocated \verb+octtree_type+ for intersecting elements.
  
\begin{lstlisting}[language=FORTRAN]
  interface query
    module procedure query_octtree_allocatable, query_octtree_pointer
  end interface query
\end{lstlisting}
    
\begin{lstlisting}[language=FORTRAN]
  subroutine query_octtree_allocatable(octtree, element_a, eles_b)
    type(octtree_type), intent(inout) :: octtree
    real, dimension(:, :), intent(in) :: element_a
    integer, dimension(:), allocatable, intent(out) :: eles_b
\end{lstlisting}
  
\begin{lstlisting}[language=FORTRAN]
  subroutine query_octtree_pointer(octtree, element_a, eles_b)
    type(octtree_type), intent(inout) :: octtree
    real, dimension(:, :), intent(in) :: element_a
    integer, dimension(:), pointer, intent(out) :: eles_b
\end{lstlisting}

\begin{description}[font=\ttfamily\bfseries,leftmargin=2.2\parindent,labelindent=1.7\parindent,noitemsep]
  \item[octtree] Octtree data structure.
  \item[element\_a] $d \times l_A$ array. Node coordinates of a mesh A element.
  \item[eles\_b] Elements of mesh B whose AABBs intersect with the AABB of the
    given mesh A element.
\end{description}

\noindent Deallocate an \verb+octtree_type+.
  
\begin{lstlisting}[language=FORTRAN]
  interface deallocate
    module procedure deallocate_octtree
  end interface deallocate
\end{lstlisting}
  
\begin{lstlisting}[language=FORTRAN]
  subroutine deallocate_octtree(octtree)
    type(octtree_type), intent(inout) :: octtree
\end{lstlisting}

\begin{description}[font=\ttfamily\bfseries,leftmargin=2.2\parindent,labelindent=1.7\parindent,noitemsep]
  \item[octtree] Octtree data structure.
\end{description}

\subsubsection{\texttt{tree\_type}}

An intersection finder for 2D and 3D meshes, which uses either the
\verb+quadtree_type+ (section \ref{sect:quadtree_query}) or \verb+octtree_type+
(section \ref{sect:octtree_query}).

~\newline
Allocate a \verb+tree_type+, which can later be queried for intersecting
elements.

\begin{lstlisting}[language=FORTRAN]
  interface allocate
    module procedure allocate_tree
  end interface allocate
\end{lstlisting}
  
\begin{lstlisting}[language=FORTRAN]
  subroutine allocate_tree(tree, positions, enlist)
    type(tree_type), intent(out) :: tree
    real, dimension(:, :), intent(in) :: positions
    integer, dimension(:, :), intent(in) :: enlist
\end{lstlisting}

\begin{description}[font=\ttfamily\bfseries,leftmargin=2.2\parindent,labelindent=1.7\parindent,noitemsep]
  \item[tree] Quadtree or octtree data structure.
  \item[positions] $d \times V_B$ array. Mesh B node coordinates.
  \item[enlist] $l_A \times E_B$ array. Mesh B element-node graph.
\end{description}

\noindent Query an allocated \verb+tree_type+ for intersecting elements.
  
\begin{lstlisting}[language=FORTRAN]
  interface query
    module procedure query_tree_allocatable, query_tree_pointer
  end interface query
\end{lstlisting}
    
\begin{lstlisting}[language=FORTRAN]  
  subroutine query_tree_allocatable(tree, element_a, eles_b)
    type(tree_type), intent(inout) :: tree
    real, dimension(:, :), intent(in) :: element_a
    integer, dimension(:), allocatable, intent(out) :: eles_b
\end{lstlisting}
  
\begin{lstlisting}[language=FORTRAN]  
  subroutine query_tree_pointer(tree, element_a, eles_b)
    type(tree_type), intent(inout) :: tree
    real, dimension(:, :), intent(in) :: element_a
    integer, dimension(:), pointer, intent(out) :: eles_b
\end{lstlisting}

\begin{description}[font=\ttfamily\bfseries,leftmargin=2.2\parindent,labelindent=1.7\parindent,noitemsep]
  \item[tree] Quadtree or octtree data structure.
  \item[element\_a] $d \times l_A$ array. Node coordinates of a mesh A element.
  \item[eles\_b] Elements of mesh B whose AABBs intersect with the AABB of the
    given mesh A element.
\end{description}

\noindent Deallocate a \verb+tree_type+.
  
\begin{lstlisting}[language=FORTRAN]
  interface deallocate
    module procedure deallocate_tree
  end interface deallocate
\end{lstlisting}
  
\begin{lstlisting}[language=FORTRAN]  
  subroutine deallocate_tree(tree)
    type(tree_type), intent(inout) :: tree
\end{lstlisting}

\begin{description}[font=\ttfamily\bfseries,leftmargin=2.2\parindent,labelindent=1.7\parindent,noitemsep]
  \item[tree] Quadtree or octtree data structure.
\end{description}

\subsubsection{\texttt{rtree\_type}}\label{sect:rtree_query}

An intersection finder for 2D and 3D meshes, using a libspatialindex 1.8.5
R${}^*$-tree\footnote{\url{https://libspatialindex.github.io/}}.

~\newline
Allocate an \verb+rtree_type+, which can later be queried for intersecting
elements.

\begin{lstlisting}[language=FORTRAN]
  interface allocate
    module procedure allocate_rtree
  end interface allocate
\end{lstlisting}
  
\begin{lstlisting}[language=FORTRAN]  
  subroutine allocate_rtree(rtree, positions, enlist)
    type(rtree_type), intent(out) :: rtree
    real(kind = c_double), dimension(:, :), intent(in) :: positions
    integer(kind = c_int), dimension(:, :), intent(in) :: enlist
\end{lstlisting}

\begin{description}[font=\ttfamily\bfseries,leftmargin=2.2\parindent,labelindent=1.7\parindent,noitemsep]
  \item[rtree] libspatialindex R${}^*$-tree data structure.
  \item[positions] $d \times V_B$ array. Mesh B node coordinates.
  \item[enlist] $l_A \times E_B$ array. Mesh B element-node graph.
\end{description}

\noindent Query an allocated \verb+rtree_type+ for intersecting elements.
  
\begin{lstlisting}[language=FORTRAN]
  interface query
    module procedure query_rtree_allocatable, query_rtree_pointer
  end interface query
\end{lstlisting}
    
\begin{lstlisting}[language=FORTRAN]    
  subroutine query_rtree_allocatable(rtree, element_a, eles_b)
    type(rtree_type), intent(inout) :: rtree
    real(kind = c_double), dimension(:, :), intent(in) :: element_a
    integer(kind = c_int), dimension(:), allocatable, intent(out) :: eles_b
\end{lstlisting}
  
\begin{lstlisting}[language=FORTRAN]      
  subroutine query_rtree_pointer(rtree, element_a, eles_b)
    type(rtree_type), intent(inout) :: rtree
    real(kind = c_double), dimension(:, :), intent(in) :: element_a
    integer(kind = c_int), dimension(:), pointer, intent(out) :: eles_b
\end{lstlisting}

\begin{description}[font=\ttfamily\bfseries,leftmargin=2.2\parindent,labelindent=1.7\parindent,noitemsep]
  \item[rtree] libspatialindex R${}^*$-tree data structure.
  \item[element\_a] $d \times l_A$ array. Node coordinates of a mesh A element.
  \item[eles\_b] Elements of mesh B whose AABBs intersect with the AABB of the
    given mesh A element.
\end{description}

\noindent Deallocate an \verb+rtree_type+.
  
\begin{lstlisting}[language=FORTRAN]
  interface deallocate
    module procedure deallocate_rtree
  end interface deallocate
\end{lstlisting}
  
\begin{lstlisting}[language=FORTRAN]  
  subroutine deallocate_rtree(rtree)
    type(rtree_type), intent(inout) :: rtree
\end{lstlisting}

\begin{description}[font=\ttfamily\bfseries,leftmargin=2.2\parindent,labelindent=1.7\parindent,noitemsep]
  \item[rtree] libspatialindex R${}^*$-tree data structure.
\end{description}

\section{Element intersection}

Given two elements, element A from mesh A and element B from mesh B, the
following routines are used to construct a simplicial mesh ``C'', which is a
mesh of their intersection (the ``local supermesh'').

The following definitions are used:
\begin{description}[leftmargin=\parindent,labelindent=\parindent]
  \item[$d$] Mesh A defines a subset $\Omega_A \subset \mathbb{R}^d$, and mesh B
    a subset $\Omega_B \subset \mathbb{R}^d$.
  \item[$l$] Number of nodes in an element.
  \item[$l_A$] Number of nodes in element A.
  \item[$l_B$] Number of nodes in element B.
  \item[$F_A$] Number of element A facets.
  \item[$F_B$] Number of element B facets.
  \item[$l_C$] Number of nodes per element in the intersection mesh C, always
               equal to $d + 1$.
  \item[$E_C$] Number of elements in the intersection mesh C.
  \item[$M_C$] An upper bound for the maximum number of elements which can be
               in any intersection mesh C.
\end{description}

\subsection{One dimension}\label{sect:1D_intersection}

Definition of $M_C$.
  
\begin{lstlisting}[language=FORTRAN] 
  integer, parameter :: interval_buf_size = 1
\end{lstlisting}

~\newline
Interval intersection.
  
\begin{lstlisting}[language=FORTRAN]    
  interface intersect_intervals
    module procedure intersect_intervals_rank_1, intersect_intervals_rank_2, &
      & intersect_intervals_rank_3
  end interface intersect_intervals
\end{lstlisting}

\begin{lstlisting}[language=FORTRAN]   
  pure subroutine intersect_intervals_rank_1(interval_a, interval_b, &
    & intervals_c, n_intervals_c)
    real, dimension(2), intent(in) :: interval_a
    real, dimension(2), intent(in) :: interval_b
    real, dimension(2), intent(out) :: intervals_c
    integer, intent(out) :: n_intervals_c
\end{lstlisting} 

\begin{description}[font=\ttfamily\bfseries,leftmargin=2.2\parindent,labelindent=1.7\parindent,noitemsep]
  \item[interval\_a] Length $2$ vector. Element A node coordinates.
  \item[interval\_b] Length $2$ vector. Element B node coordinates.
  \item[interval\_c] Length $2$ vector. Intersection mesh C node coordinates.
  \item[n\_intervals\_c] Number of elements in mesh C. Equal to $1$ if elements
    A and B intersect, and $0$ otherwise.
\end{description}
  
\begin{lstlisting}[language=FORTRAN]   
  pure subroutine intersect_intervals_rank_2(interval_a, interval_b, &
    & intervals_c, n_intervals_c)
    real, dimension(2), intent(in) :: interval_a
    real, dimension(2), intent(in) :: interval_b
    real, dimension(2, interval_buf_size), intent(out) :: intervals_c
    integer, intent(out) :: n_intervals_c
\end{lstlisting} 

\begin{description}[font=\ttfamily\bfseries,leftmargin=2.2\parindent,labelindent=1.7\parindent,noitemsep]
  \item[interval\_a] Length $2$ vector. Element A node coordinates.
  \item[interval\_b] Length $2$ vector. Element B node coordinates.
  \item[interval\_c] $2 \times M_C$ array. Intersection mesh C node coordinates.
  \item[n\_intervals\_c] Number of elements in mesh C. Equal to $1$ if elements
    A and B intersect, and $0$ otherwise. $M_C = 1$.
\end{description}
  
\begin{lstlisting}[language=FORTRAN]  
  pure subroutine intersect_intervals_rank_3(interval_a, interval_b, &
    & intervals_c, n_intervals_c)
    real, dimension(1, 2), intent(in) :: interval_a
    real, dimension(1, 2), intent(in) :: interval_b
    real, dimension(1, 2, interval_buf_size), intent(out) :: intervals_c
    integer, intent(out) :: n_intervals_c
\end{lstlisting} 

\begin{description}[font=\ttfamily\bfseries,leftmargin=2.2\parindent,labelindent=1.7\parindent,noitemsep]
  \item[interval\_a] $1 \times 2$ array. Element A node coordinates.
  \item[interval\_b] $1 \times 2$ vector. Element B node coordinates.
  \item[interval\_c] $1 \times 2 \times M_C$ array. Intersection mesh C node
    coordinates. $M_C = 1$.
  \item[n\_intervals\_c] Number of elements in mesh C. Equal to $1$ if elements
    A and B intersect, and $0$ otherwise.
\end{description}

\noindent Interval length.
  
\begin{lstlisting}[language=FORTRAN]  
  interface interval_size
    module procedure interval_size_rank_1, interval_size_rank_2
  end interface interval_size
\end{lstlisting} 
  
\begin{lstlisting}[language=FORTRAN]  
  pure function interval_size_rank_1(interval) result(size)
    real, dimension(2), intent(in) :: interval
    real :: size
\end{lstlisting} 

\begin{description}[font=\ttfamily\bfseries,leftmargin=2.2\parindent,labelindent=1.7\parindent,noitemsep]
  \item[interval] Length $2$ vector. Element node coordinates.
  \item[size] Length of the element.
\end{description}
  
\begin{lstlisting}[language=FORTRAN]    
  pure function interval_size_rank_2(interval) result(size)
    real, dimension(1, 2), intent(in) :: interval
    real :: size
\end{lstlisting} 

\begin{description}[font=\ttfamily\bfseries,leftmargin=2.2\parindent,labelindent=1.7\parindent,noitemsep]
  \item[interval] $1 \times 2$ array. Element node coordinates.
  \item[size] Length of the element.
\end{description}

\subsection{Two dimensions}\label{sect:2D_intersection}

Intersection of two-dimensional convex polygons using the \citet{sutherland1974}
algorithm.

~\newline
\verb+triangle_type+ data type, used to define triangles.

\begin{lstlisting}[language=FORTRAN]    
  type tri_type
    real, dimension(2, 3) :: v
  end type tri_type
\end{lstlisting} 

\begin{description}[font=\ttfamily\bfseries,leftmargin=2.2\parindent,labelindent=1.7\parindent,noitemsep]
  \item[v] $d \times l$ array. Triangle node coordinates.
\end{description}

\noindent \verb+line_type+ data type, used to define the lines which constitute
element facets.

\begin{lstlisting}[language=FORTRAN]   
  type line_type
    real, dimension(2) :: normal
    real, dimension(2) :: point
  end type line_type
\end{lstlisting} 

\begin{description}[font=\ttfamily\bfseries,leftmargin=2.2\parindent,labelindent=1.7\parindent,noitemsep]
  \item[normal] Length $d$ vector. Components of an outward pointing facet
    normal. Need not have unit norm.
  \item[point] Length $d$ vector. Coordinates of a point on the facet.
\end{description}

\noindent Definition of $M_C$ for triangle-triangle intersection.
  
\begin{lstlisting}[language=FORTRAN] 
  integer, parameter :: tri_buf_size = 22
\end{lstlisting}

\noindent Definition of $M_C$ for triangle-polygon intersection.
  
\begin{lstlisting}[language=FORTRAN]
  interface max_n_tris_c
    module procedure max_n_tris_c_tri
  end interface max_n_tris_c
\end{lstlisting}

\begin{lstlisting}[language=FORTRAN]
  pure function max_n_tris_c_tri(n_lines_b) result(max_n_tris_c)
    integer, intent(in) :: n_lines_b
    integer :: max_n_tris_c
\end{lstlisting} 

\begin{description}[font=\ttfamily\bfseries,leftmargin=2.2\parindent,labelindent=1.7\parindent,noitemsep]
  \item[n\_lines\_b] $F_B$, assuming that element A is a triangle.
  \item[max\_n\_tris\_c] $M_C = 3 \left( 2^{F_B} \right) - 2$.
\end{description}

\noindent Definition of $M_C$ for polygon-polygon intersection.

\begin{lstlisting}[language=FORTRAN]
  interface max_n_tris_c
    module procedure max_n_tris_c_poly
  end interface max_n_tris_c
\end{lstlisting} 

\begin{lstlisting}[language=FORTRAN]
  pure function max_n_tris_c_poly(n_lines_a, n_lines_b) result(max_n_tris_c)
    integer, intent(in) :: n_lines_a
    integer, intent(in) :: n_lines_b
    integer :: max_n_tris_c
\end{lstlisting} 

\begin{description}[font=\ttfamily\bfseries,leftmargin=2.2\parindent,labelindent=1.7\parindent,noitemsep]
  \item[n\_lines\_a] $F_A$.
  \item[n\_lines\_b] $F_B$.
  \item[max\_n\_tris\_c] $M_C = F_A \left( 2^{F_B} \right) - 2$.
\end{description}

\noindent Obtain a \verb+line_type+ representation of facets.

\begin{lstlisting}[language=FORTRAN]
  interface get_lines
    module procedure get_lines_tri, get_lines_poly
  end interface get_lines
\end{lstlisting} 

\begin{lstlisting}[language=FORTRAN]
  pure function get_lines_tri(tri) result(lines)
    type(tri_type), intent(in) :: tri
    type(line_type), dimension(3) :: lines
\end{lstlisting}

\begin{description}[font=\ttfamily\bfseries,leftmargin=2.2\parindent,labelindent=1.7\parindent,noitemsep]
  \item[tri] A triangle.
  \item[lines] The three facets of the triangle.
\end{description}

\begin{lstlisting}[language=FORTRAN]
  pure function get_lines_poly(poly) result(lines)
    real, dimension(:, :), intent(in) :: poly    
    type(line_type), dimension(size(poly, 2)) :: lines
\end{lstlisting}

\begin{description}[font=\ttfamily\bfseries,leftmargin=2.2\parindent,labelindent=1.7\parindent,noitemsep]
  \item[poly] $d \times l$ array. Coordinates of the polygon nodes, in either
    clockwise or anti-clockwise order. Must define a convex polygon.
  \item[lines] Length $l$ array. The facets of the polygon.
\end{description}

\noindent Triangle-triangle intersection.

\begin{lstlisting}[language=FORTRAN]
  interface intersect_tris
    module procedure intersect_tris_real, intersect_tris_tri
  end interface intersect_tris
\end{lstlisting}

\begin{lstlisting}[language=FORTRAN]
  subroutine intersect_tris_real(tri_a, tri_b, tris_c, n_tris_c)
    real, dimension(2, 3), intent(in) :: tri_a
    real, dimension(2, 3), intent(in) :: tri_b
    real, dimension(2, 3, tri_buf_size), intent(out) :: tris_c
    integer, intent(out) :: n_tris_c
\end{lstlisting}

\begin{description}[font=\ttfamily\bfseries,leftmargin=2.2\parindent,labelindent=1.7\parindent,noitemsep]
  \item[tri\_a] $d \times l_A$ array. Element A node coordinates. Element A
    must be a triangle.
  \item[tri\_b] $d \times l_B$ array. Element B node coordinates. Element B
    must be a triangle.
  \item[tris\_c] $d \times l_C \times M_C$ array. Mesh C node coordinates.
  \item[n\_tris\_c] $E_C$. $M_C = 22$.
\end{description}

\begin{lstlisting}[language=FORTRAN]
  subroutine intersect_tris_tri(tri_a, tri_b, tris_c, n_tris_c)
    type(tri_type), intent(in) :: tri_a
    type(tri_type), intent(in) :: tri_b
    type(tri_type), dimension(tri_buf_size), intent(out) :: tris_c
    integer, intent(out) :: n_tris_c
\end{lstlisting}

\begin{description}[font=\ttfamily\bfseries,leftmargin=2.2\parindent,labelindent=1.7\parindent,noitemsep]
  \item[tri\_a] Element A.
  \item[tri\_b] Element B.
  \item[tris\_c] Length $M_C$ vector. Triangles in mesh C. $M_C = 22$.
  \item[n\_tris\_c] $E_C$.
\end{description}

\noindent Triangle-quadrilateral intersection.

\begin{lstlisting}[language=FORTRAN]
  subroutine intersect_tri_quad(tri_a, quad_b, tris_c, n_tris_c)
    real, dimension(:, :), intent(in) :: tri_a
    real, dimension(:, :), intent(in) :: quad_b
    real, dimension(:, :, :), intent(inout) :: tris_c
    integer, intent(out) :: n_tris_c
\end{lstlisting}

\begin{description}[font=\ttfamily\bfseries,leftmargin=2.2\parindent,labelindent=1.7\parindent,noitemsep]
  \item[tri\_a] $d \times l_A$ array. Element A node coordinates. Element A
    must be a triangle.
  \item[quad\_b] $d \times l_B$ array. Element B node coordinates. Element B
    must be a convex quadrilateral with the node coordinates in either clockwise
    or anti-clockwise order.
  \item[tris\_c] $d \times l_C \times M_C$ array. Mesh C node coordinates.
    $M_C = 46$.
  \item[n\_tris\_c] $E_C$.
\end{description}

\noindent Quadrilateral-quadrilateral intersection.

\begin{lstlisting}[language=FORTRAN]
  subroutine intersect_quads(quad_a, quad_b, tris_c, n_tris_c)
    real, dimension(:, :), intent(in) :: quad_a
    real, dimension(:, :), intent(in) :: quad_b
    real, dimension(:, :, :), intent(inout) :: tris_c
    integer, intent(out) :: n_tris_c
\end{lstlisting}

\begin{description}[font=\ttfamily\bfseries,leftmargin=2.2\parindent,labelindent=1.7\parindent,noitemsep]
  \item[quad\_a] $d \times l_A$ array. Element A node coordinates. Element A
    must be a convex quadrilateral with the node coordinates in either clockwise
    or anti-clockwise order.
  \item[quad\_b] $d \times l_B$ array. Element B node coordinates. Element B
    must be a convex quadrilateral with the node coordinates in either clockwise
    or anti-clockwise order.
  \item[tris\_c] $d \times l_C \times M_C$ array. Mesh C node coordinates.
    $M_C = 62$.
  \item[n\_tris\_c] $E_C$.
\end{description}

\noindent Triangle-polygon intersection.
  
\begin{lstlisting}[language=FORTRAN]
  interface intersect_polys
    module procedure intersect_tri_lines
  end interface intersect_polys
\end{lstlisting}

\begin{lstlisting}[language=FORTRAN]
  subroutine intersect_tri_lines(tri_a, lines_b, tris_c, n_tris_c, area_b, work)
    type(tri_type), intent(in) :: tri_a
    type(line_type), dimension(:), intent(in) :: lines_b
    type(tri_type), dimension(:), intent(out) :: tris_c
    integer, intent(out) :: n_tris_c
    real, optional, intent(in) :: area_b
    real, dimension(:, :, :), target, optional, intent(out) :: work
\end{lstlisting}

\begin{description}[font=\ttfamily\bfseries,leftmargin=2.2\parindent,labelindent=1.7\parindent,noitemsep]
  \item[tri\_a] Element A.
  \item[lines\_b] Element B facets. Must be facets for a convex polygon.
  \item[tris\_c] Length $M_C$ vector. Triangles in mesh C.
  \item[n\_tris\_c] $E_C$.
  \item[area\_b] Area of element B. Used to discard near-degenerate elements in
    mesh C.
  \item[work] $2 \times M_C \times 2$ array. Working memory.
\end{description}

\noindent Polygon-polygon intersection.
  
\begin{lstlisting}[language=FORTRAN]
  interface intersect_polys
    module procedure intersect_polys_real, intersect_polys_lines
  end interface intersect_polys
\end{lstlisting}
  
\begin{lstlisting}[language=FORTRAN]
  subroutine intersect_polys_real(poly_a, poly_b, tris_c, n_tris_c, &
    & area_a, area_b, work)
    real, dimension(:, :), intent(in) :: poly_a
    real, dimension(:, :), intent(in) :: poly_b
    type(tri_type), dimension(:), intent(out) :: tris_c
    integer, intent(out) :: n_tris_c
    real, optional, intent(in) :: area_a
    real, optional, intent(in) :: area_b
    real, dimension(:, :, :), target, optional, intent(out) :: work
\end{lstlisting}

\begin{description}[font=\ttfamily\bfseries,leftmargin=2.2\parindent,labelindent=1.7\parindent,noitemsep]
  \item[poly\_a] $d \times l_A$ array. Element A node coordinates. The node
    coordinates must be in either clockwise or anti-clockwise order, and must
    define a convex polygon.
  \item[poly\_b] $d \times l_B$ array. Element B node coordinates. The node
    coordinates must be in either clockwise or anti-clockwise order, and must
    define a convex polygon.
  \item[tris\_c] Length $M_C$ vector. Triangles in mesh C.
  \item[n\_tris\_c] $E_C$.
  \item[area\_a] Area of element A. Used to discard near-degenerate elements in
    mesh C.
  \item[area\_b] Area of element B. Used to discard near-degenerate elements in
    mesh C.
  \item[work] $2 \times M_C \times 2$ array. Working memory.
\end{description}

\begin{lstlisting}[language=FORTRAN]
  subroutine intersect_polys_lines(poly_a, lines_b, tris_c, n_tris_c, &
    & area_a, area_b, work)
    real, dimension(:, :), intent(in) :: poly_a
    type(line_type), dimension(:), intent(in) :: lines_b
    type(tri_type), dimension(:), intent(out) :: tris_c
    integer, intent(out) :: n_tris_c
    real, optional, intent(in) :: area_a
    real, optional, intent(in) :: area_b
    real, dimension(:, :, :), target, optional, intent(out) :: work
\end{lstlisting}

\begin{description}[font=\ttfamily\bfseries,leftmargin=2.2\parindent,labelindent=1.7\parindent,noitemsep]
  \item[poly\_a] $d \times l_A$ array. Element A node coordinates. The node
    coordinates must be in either clockwise or anti-clockwise order, and must
    define a convex polygon.
  \item[lines\_b] $d \times l_B$ array. Element B facets. Must be facets for a
    convex polygon.
  \item[tris\_c] Length $M_C$ vector. Triangles in mesh C.
  \item[n\_tris\_c] $E_C$.
  \item[area\_a] Area of element A. Used to discard near-degenerate elements in
    mesh C.
  \item[area\_b] Area of element B. Used to discard near-degenerate elements in
    mesh C.
  \item[work] $2 \times M_C \times 2$ array. Working memory.
\end{description}

\noindent Triangle area.

\begin{lstlisting}[language=FORTRAN]
  interface triangle_area
    module procedure triangle_area_real, triangle_area_tri
  end interface triangle_area
\end{lstlisting}

\begin{lstlisting}[language=FORTRAN]
  pure function triangle_area_real(tri) result(area)
    real, dimension(2, 3), intent(in) :: tri
    real :: area
\end{lstlisting}

\begin{description}[font=\ttfamily\bfseries,leftmargin=2.2\parindent,labelindent=1.7\parindent,noitemsep]
  \item[tri] $d \times l$ array. Triangle node coordinates.
  \item[area] Area of the triangle.
\end{description}

\begin{lstlisting}[language=FORTRAN]
  pure function triangle_area_tri(tri) result(area)
    type(tri_type), intent(in) :: tri
    real :: area
\end{lstlisting}

\begin{description}[font=\ttfamily\bfseries,leftmargin=2.2\parindent,labelindent=1.7\parindent,noitemsep]
  \item[tri] A triangle.
  \item[area] Area of the triangle.
\end{description}

\subsection{Three dimensions}\label{sect:3D_intersection}

Intersection of three-dimensional convex polyhedra using the ``plane-at-a-time
clipping'' algorithm described in \citet[][section 2.4.3]{eberly2007}.

~\newline
\verb+tet_type+ data type, used to define tetrahedra.

\begin{lstlisting}[language=FORTRAN]
  type tet_type
    real, dimension(3, 4) :: v
    integer, dimension(4) :: colours = -1
  end type tet_type
\end{lstlisting}

\begin{description}[font=\ttfamily\bfseries,leftmargin=2.2\parindent,labelindent=1.7\parindent,noitemsep]
  \item[v] $d \times l$ array. Tetrahedron node coordinates.
  \item[colours] Facet colouring. Used to colour facets of an intersection mesh
    C, using the colours defined in a tetrahedral element A. \verb+colours(i)+
    indicates the colour of the facet opposite to the node with coordinate
    \verb+v(:, i)+. A negative or zero value indicates no colour.
\end{description}

\noindent \verb+plane_type+ data type, used to define the planes which
constitute element facets.

\begin{lstlisting}[language=FORTRAN]
  type plane_type
    real, dimension(3) :: normal
    real :: c
  end type plane_type
\end{lstlisting}

\begin{description}[font=\ttfamily\bfseries,leftmargin=2.2\parindent,labelindent=1.7\parindent,noitemsep]
  \item[normal] Length $d$. Components of an outward pointing facet normal, with
    unit norm.
  \item[c] Distance of the origin from the plane.
\end{description}

\noindent Definition of $M_C$ for tetrahedron-tetrahedron intersection.

\begin{lstlisting}[language=FORTRAN]
  integer, parameter :: tet_buf_size = 81
\end{lstlisting}

\noindent Definition of $M_C$ for tetrahderon-polyhedron intersection.

\begin{lstlisting}[language=FORTRAN]
  pure function max_n_tets_c(n_planes_b)
    integer, intent(in) :: n_planes_b    
    integer :: max_n_tets_c
\end{lstlisting}

\begin{description}[font=\ttfamily\bfseries,leftmargin=2.2\parindent,labelindent=1.7\parindent,noitemsep]
  \item[n\_planes\_b] $F_B$, assuming that element B is a tetrahedron.
  \item[max\_n\_tets\_c] $M_C = 3^{F_B}$.
\end{description}

\noindent Obtain a \verb+plane_type+ representation of facets.

\begin{lstlisting}[language=FORTRAN]
  interface get_planes
    module procedure get_planes_tet
  end interface get_planes
\end{lstlisting}
  
\begin{lstlisting}[language=FORTRAN]
  pure function get_planes_tet(tet) result(planes)
    type(tet_type), intent(in) :: tet
    type(plane_type), dimension(4) :: planes
\end{lstlisting}

\begin{description}[font=\ttfamily\bfseries,leftmargin=2.2\parindent,labelindent=1.7\parindent,noitemsep]
  \item[tet] A tetrahedron.
  \item[plane] The four facets of the tetrahedron.
\end{description}

\noindent Tetrahedron-tetrahderon intersection.

\begin{lstlisting}[language=FORTRAN]
  interface intersect_tets
    module procedure intersect_tets_real, intersect_tets_tet
  end interface intersect_tets
\end{lstlisting}

\begin{lstlisting}[language=FORTRAN]
  subroutine intersect_tets_real(tet_a, tet_b, tets_c, n_tets_c)
    real, dimension(3, 4), intent(in) :: tet_a
    real, dimension(3, 4), intent(in) :: tet_b
    real, dimension(3, 4, tet_buf_size), intent(inout) :: tets_c
    integer, intent(out) :: n_tets_c
\end{lstlisting}

\begin{description}[font=\ttfamily\bfseries,leftmargin=2.2\parindent,labelindent=1.7\parindent,noitemsep]
  \item[tet\_a] $d \times l_A$ array. Element A node coordinates. Element A must
    be a tetrahedron.
  \item[tet\_b] $d \times l_B$ array. Element B node coordinates. Element B must
    be a tetrahedron.
  \item[tets\_c] $d \times l_C \times M_C$ array. Mesh C node coordinates.
    $M_C = 81$.
  \item[n\_tets\_c] $E_C$.
\end{description}

\begin{lstlisting}[language=FORTRAN]
  subroutine intersect_tets_tet(tet_a, tet_b, tets_c, n_tets_c)
    type(tet_type), intent(in) :: tet_a
    type(tet_type), intent(in) :: tet_b
    type(tet_type), dimension(tet_buf_size), intent(inout) :: tets_c
    integer, intent(out) :: n_tets_c
\end{lstlisting}

\begin{description}[font=\ttfamily\bfseries,leftmargin=2.2\parindent,labelindent=1.7\parindent,noitemsep]
  \item[tet\_a] Element A.
  \item[tet\_b] Element B.
  \item[tets\_c] Length $M_C$ vector. Tetrahedra in mesh C. $M_C = 81$.
  \item[n\_tets\_c] $E_C$.
\end{description}

\noindent Tetrahedron-hexahedron intersection.

\begin{lstlisting}[language=FORTRAN]
  subroutine intersect_tet_hex(tet_a, hex_b, tets_c, n_tets_c)
    real, dimension(:, :), intent(in) :: tet_a
    real, dimension(:, :), intent(in) :: hex_b
    real, dimension(:, :, :), intent(inout) :: tets_c
    integer, intent(out) :: n_tets_c
\end{lstlisting}

\begin{description}[font=\ttfamily\bfseries,leftmargin=2.2\parindent,labelindent=1.7\parindent,noitemsep]
  \item[tet\_a] $d \times l_A$ array. Element A node coordinates. Element A must
    be a tetrahedron.
  \item[hex\_b] $d \times l_B$ array. Element B node coordinates. Element B must
    be a convex hexahedron with Gmsh 2.12 node
    ordering\footnote{See \url{http://gmsh.info/doc/texinfo/gmsh.html\#Node-ordering}.}.
  \item[tets\_c] $d \times l_C \times M_C$ array. Mesh C node coordinates.
    $M_C = 729$.
  \item[n\_tets\_c] $E_C$.
\end{description}

\noindent Hexahedron-hexahedron intersection. Divides the convex hexahedron A
into five tetraheda \citep[as in][figure 2.29]{wenger2013}, and then uses
\verb+intersect_tet_hex+ to intersect these with the convex hexahedron B.

\begin{lstlisting}[language=FORTRAN]
  subroutine intersect_hexes(hex_a, hex_b, tets_c, n_tets_c)
    real, dimension(:, :), intent(in) :: hex_a
    real, dimension(:, :), intent(in) :: hex_b
    real, dimension(:, :, :), intent(inout) :: tets_c
    integer, intent(out) :: n_tets_c
\end{lstlisting}

\begin{description}[font=\ttfamily\bfseries,leftmargin=2.2\parindent,labelindent=1.7\parindent,noitemsep]
  \item[hex\_a, hex\_b] $d \times l_A$ arrays. Element A and B node coordinates.
    Each element must be a convex hexahedron with Gmsh 2.12 node ordering.
  \item[tets\_c] $d \times l_C \times M_C$ array. Mesh C node coordinates.
    $M_C = 3,645$.
  \item[n\_tets\_c] $E_C$.
\end{description}

\noindent Tetrahedron-polyhedron intersection.

\begin{lstlisting}[language=FORTRAN]
  interface intersect_tets
    module procedure intersect_tets_planes
  end interface intersect_tets
\end{lstlisting}
  
\begin{lstlisting}[language=FORTRAN]
  subroutine intersect_tets_planes(tet_a, planes_b, tets_c, n_tets_c, vol_b, work)
    type(tet_type), intent(in) :: tet_a
    type(plane_type), dimension(:), intent(in)  :: planes_b
    type(tet_type), dimension(:), intent(inout) :: tets_c
    integer, intent(out) :: n_tets_c
    real, optional, intent(in) :: vol_b
    type(tet_type), dimension(:), target, optional, intent(inout) :: work
\end{lstlisting}

\begin{description}[font=\ttfamily\bfseries,leftmargin=2.2\parindent,labelindent=1.7\parindent,noitemsep]
  \item[tet\_a] Element A.
  \item[planes\_b] Element B facets. Must be facets for a convex polyhedron.
  \item[tets\_c] $d \times l_C \times M_C$ array. Mesh C node coordinates.
  \item[n\_tets\_c] $E_C$.
  \item[vol\_b] Volume of element $B$. Used to discard near-degenerate elements
    in mesh C.
  \item[work] Length $M_C$ vector. Working memory.
\end{description}

\noindent Tetrahedron volume.

\begin{lstlisting}[language=FORTRAN]
  interface tetrahedron_volume
    module procedure tetrahedron_volume_real, tetrahedron_volume_tet
  end interface tetrahedron_volume
\end{lstlisting}

\begin{lstlisting}[language=FORTRAN]
  pure function tetrahedron_volume_real(tet) result(volume)
    real, dimension(3, 4), intent(in) :: tet
    real :: volume
\end{lstlisting}

\begin{description}[font=\ttfamily\bfseries,leftmargin=2.2\parindent,labelindent=1.7\parindent,noitemsep]
  \item[tet] $d \times l$ array. Tetrahedron node coordinates.
  \item[volume] Volume of the tetrahedron.
\end{description}

\begin{lstlisting}[language=FORTRAN]
  pure function tetrahedron_volume_tet(tet) result(volume)
    type(tet_type), intent(in) :: tet
    real :: volume
\end{lstlisting}

\begin{description}[font=\ttfamily\bfseries,leftmargin=2.2\parindent,labelindent=1.7\parindent,noitemsep]
  \item[tet] A tetrahedron.
  \item[volume] Volume of the tetrahedron.
\end{description}

\subsection{General dimensions}\label{sect:nD_intersection}

Definition of $M_C$ for simplex-simplex intersection in one, two, and three
dimensions.

\begin{lstlisting}[language=FORTRAN]
  interface max_n_simplices_c
    module procedure max_n_simplices_c_simplices
  end interface max_n_simplices_c
\end{lstlisting}

\begin{lstlisting}[language=FORTRAN]
  pure function max_n_simplices_c_simplices(dim) result(size)
    integer, intent(in) :: dim
    integer :: size
\end{lstlisting}

\begin{description}[font=\ttfamily\bfseries,leftmargin=2.2\parindent,labelindent=1.7\parindent,noitemsep]
  \item[dim] Dimension $d$.
  \item[size] $M_C$. Equal to $1$ for the interval-interval case, $22$ for the
    triangle-triangle case, and $81$ for the tetrahedron-tetrahedron case.
\end{description}

\noindent Definition of $M_C$ for simplex-simplex, simplex-hypercube, and
hypercube-hypercube intersection in one, two, and three dimensions.
  
\begin{lstlisting}[language=FORTRAN]
  interface max_n_simplices_c
    module procedure max_n_simplices_c_elements
  end interface max_n_simplices_c
\end{lstlisting}

\begin{lstlisting}[language=FORTRAN]
  pure function max_n_simplices_c_elements(dim, loc_a, loc_b) result(size)
    integer, intent(in) :: dim
    integer, intent(in) :: loc_a
    integer, intent(in) :: loc_b
    integer :: size
\end{lstlisting}

\begin{description}[font=\ttfamily\bfseries,leftmargin=2.2\parindent,labelindent=1.7\parindent,noitemsep]
  \item[dim] Dimension $d$. Must be equal to $1$, $2$, or $3$.
  \item[loc\_a] $l_A$.
  \item[loc\_b] $l_B$.
  \item[size] $M_C$. Equal to \verb+max_n_simplices_c(d)+ for the
    simplex-simplex case, $46$ for the triangle-quadrilateral case, $62$ for the
    quadrilateral-quadrilateral case, $729$ for the tetrahedron-hexahedron case,
    and $3,645$ for the hexahedron-hexahedron case.
\end{description}

\noindent Simplex-simplex intersection in one, two, and three dimensions.

\begin{lstlisting}[language=FORTRAN]
  subroutine intersect_simplices(simplex_a, simplex_b, &
    & simplices_c, n_simplices_c)
    real, dimension(:, :), intent(in) :: simplex_a
    real, dimension(:, :), intent(in) :: simplex_b
    real, dimension(:, :, :), intent(inout) :: simplices_c
    integer, intent(out) :: n_simplices_c
\end{lstlisting}

\begin{description}[font=\ttfamily\bfseries,leftmargin=2.2\parindent,labelindent=1.7\parindent,noitemsep]
  \item[simplex\_a, simplex\_b] $d \times l_A$ arrays. Element A and B node
    coordinates. Each element must be a one, two, or three dimensional simplex.
  \item[simplex\_c] $d \times l_C \times M_C$ array. Mesh C node coordinates.
  \item[n\_simplices\_c] $E_C$.
\end{description}

\noindent Simplex-simplex, simplex-hypercube, and hypercube-hypercube
intersection in one, two, and three dimensions.

\begin{lstlisting}[language=FORTRAN]
  subroutine intersect_elements(element_a, element_b, elements_c, nelements_c)
    real, dimension(:, :), intent(in) :: element_a
    real, dimension(:, :), intent(in) :: element_b
    real, dimension(:, :, :), intent(inout) :: elements_c
    integer, intent(out) :: nelements_c
\end{lstlisting}

\begin{description}[font=\ttfamily\bfseries,leftmargin=2.2\parindent,labelindent=1.7\parindent,noitemsep]
  \item[element\_a, element\_b] $d \times l_A$ arrays. Element A and B node
    coordinates. Each element must be a one, two, or three dimensional simplex
    or convex hypercube. Quadrilaterals must have nodes ordered in a clockwise
    or anti-clockwise order, and hexahedra must use Gmsh 2.12 node ordering.
  \item[elements\_c] $d \times l_C \times M_C$ array. Mesh C node coordinates.
  \item[n\_simplices\_c] $E_C$.
\end{description}

\noindent Simplex volume in one, two, and three dimensions.

\begin{lstlisting}[language=FORTRAN]
  pure function simplex_volume(simplex) result(volume)
    real, dimension(:, :), intent(in) :: simplex
    real :: volume
\end{lstlisting}

\begin{description}[font=\ttfamily\bfseries,leftmargin=2.2\parindent,labelindent=1.7\parindent,noitemsep]
  \item[simplex] $d \times l$ array. Simplex node coordinates. Must be
    coordinates for a one, two, or three dimensional simplex.
  \item[volume] Simplex volume.
\end{description}

\section{Parallelisation}

libsupermesh provides a high-level interface for performing parallel supermesh
calculations, for the case where the domain decomposition of the two meshes A
and B do not match. 

In the following the local mesh A refers to the local submesh (or partition) of
mesh A stored on the current MPI process. Similarly the local mesh B refers to
the local submesh (or partition) of mesh B stored on the current MPI process.
The received mesh B refers to a submesh of mesh B received from a different MPI
process.

Candidate intersection identification is performed using the libspatialindex
R${}^*$-tree described in section \ref{sect:rtree_query}, and element
intersections are performed using the \verb+intersect_elements+ routine
described in section \ref{sect:nD_intersection}. Parallel communication is
performed using MPI.

The following definitions are used:
\begin{description}[leftmargin=\parindent,labelindent=\parindent]
  \item[$d$] Mesh A defines a subset $\Omega_A \subset \mathbb{R}^d$, and mesh B
    a subset $\Omega_B \subset \mathbb{R}^d$.
  \item[$V_A$] Number of nodes in the local mesh A.
  \item[$V_B$] Number of nodes in the local mesh B.
  \item[$E_A$] Number of elements in the local mesh A.
  \item[$E_B$] Number of elements in the local mesh B.
  \item[$V_B^*$] Number of nodes in a received mesh B.
  \item[$E_B^*$] Number of elements in a received mesh B.
  \item[$l_C$] Number of nodes per element in the intersection mesh C, always
               equal to $d + 1$.
  \item[$E_C$] Number of elements in the intersection mesh C.
\end{description}

\subsection{Algorithm}

The algorithm used is as follows:
\begin{enumerate}
  \item Communicate the AABBs of all mesh A partitions and all mesh B partitions
        using all-to-all communication
  \item For each mesh A partition whose AABB intersects with the local mesh B
        AABB
  \begin{enumerate}
    \item Identify local mesh B elements whose AABBs intersect with the AABB of
          the mesh A partition
    \item \label{alg:comm} Obtain data associated with these elements, and
          communicate these data via point-to-point communication
  \end{enumerate}
  \item Construct the intersection mesh for each pair of intersecting local
        mesh A and local mesh B elements
  \item Perform calculations on these intersection meshes
  \item For each mesh B partition whose AABB intersects with the local mesh A
        AABB
  \begin{enumerate}
    \item Unpack data communicated in step \ref{alg:comm}
    \item Construct the intersection mesh for each pair of intersecting local
          mesh A and received mesh B elements
    \item Perform calculations on these intersection meshes
  \end{enumerate}
\end{enumerate}  

\subsection{\texttt{parallel\_supermesh}}

The parallel supermeshing interface accepts the local meshes, local mesh
element ownership information, and three callbacks. The interface optionally
accepts an MPI communicator.

\begin{lstlisting}[language=FORTRAN]
  subroutine parallel_supermesh(positions_a, enlist_a, ele_owner_a, &
    & positions_b, enlist_b, ele_owner_b, &
    & pack_data_b, unpack_data_b, intersection_calculation, &
    & comm)
    real, dimension(:, :), intent(in) :: positions_a
    integer, dimension(:, :), intent(in) :: enlist_a
    integer, dimension(:), intent(in) :: ele_owner_a
    real, dimension(:, :), intent(in) :: positions_b
    integer, dimension(:, :), intent(in) :: enlist_b
    integer, dimension(:), intent(in) :: ele_owner_b
    integer, optional, intent(in) :: comm
    
    interface
      subroutine pack_data_b(nodes_b, eles_b, data_b)
        integer, dimension(:), intent(in) :: nodes_b
        integer, dimension(:), intent(in) :: eles_b
        integer(kind = c_int8_t), dimension(:), allocatable, intent(out) :: &
          & data_b
      end subroutine pack_data_b

      subroutine unpack_data_b(nnodes_b, nelements_b, data_b)
        integer, intent(in) :: nnodes_b
        integer, intent(in) :: nelements_b
        integer(kind = c_int8_t), dimension(:), intent(in) :: data_b
      end subroutine unpack_data_b
      
      subroutine intersection_calculation(positions_a, positions_b, &
        & positions_c, nodes_b, ele_a, ele_b, local)
        real, dimension(:, :), intent(in) :: positions_a
        real, dimension(:, :), intent(in) :: positions_b
        real, dimension(:, :, :), intent(in) :: positions_c
        integer, dimension(:), intent(in) :: nodes_b
        integer, intent(in) :: ele_a
        integer, intent(in) :: ele_b
        logical, intent(in) :: local
      end subroutine intersection_calculation
    end interface
\end{lstlisting}

\begin{description}[font=\ttfamily\bfseries,leftmargin=2.2\parindent,labelindent=1.7\parindent,noitemsep]
  \item[positions\_a] $d \times V_A$ array. Local mesh A node coordinates.
  \item[enlist\_a] $l_A \times E_A$ array. Local mesh A element-node graph.
  \item[ele\_owner\_a] Length $E_A$ vector. MPI rank (indexed from zero) of the
    process which owns each element in the local mesh A.
  \item[positions\_b] $d \times V_B$ array. Local mesh B node coordinates.
  \item[enlist\_b] $l_B \times E_B$ array. Local mesh B element-node graph.
  \item[ele\_owner\_b] Length $E_B$ vector. MPI rank (indexed from zero) of the
    process which owns each element in the local mesh B.
  \item[pack\_data\_b, unpack\_data\_b, intersection\_calculation] Callbacks
    used to pack and unpack mesh B data, and to perform calculations on
    intersection meshes.
  \item[comm] MPI communicator.
\end{description}

\subsection{\texttt{callbacks}}

\noindent The first callback is used to pack mesh B data for communication.

\begin{lstlisting}[language=FORTRAN]
  subroutine pack_data_b(nodes_b, eles_b, data_b)
    integer, dimension(:), intent(in) :: nodes_b
    integer, dimension(:), intent(in) :: eles_b
    integer(kind = c_int8_t), dimension(:), allocatable, intent(out) :: data_b
\end{lstlisting}

\begin{description}[font=\ttfamily\bfseries,leftmargin=2.2\parindent,labelindent=1.7\parindent,noitemsep]
  \item[nodes\_b] Number of mesh B nodes to be communicated.
  \item[eles\_b] The mesh B elements to be communicated.
  \item[data\_b] Mesh B data to be communicated. This should consist of all
    data which will later be required by the \verb+intersection_calculation+
    callback, excluding the mesh coordinates themselves.
\end{description}

\noindent The second callpack unpacks these data.

\begin{lstlisting}[language=FORTRAN]
  subroutine unpack_data_b(nnodes_b, nelements_b, data_b)
    integer, intent(in) :: nnodes_b
    integer, intent(in) :: nelements_b
    integer(kind = c_int8_t), dimension(:), intent(in) :: data_b
\end{lstlisting}

\begin{description}[font=\ttfamily\bfseries,leftmargin=2.2\parindent,labelindent=1.7\parindent,noitemsep]
  \item[nodes\_b] $V_B^*$.
  \item[eles\_b] $E_B^*$.
  \item[data\_b] Received mesh B data, previously packed on a different process
    using \verb+pack_data_b+.
\end{description}

\noindent The final callback performs calculations on each intersection mesh.
      
\begin{lstlisting}[language=FORTRAN]
  subroutine intersection_calculation(positions_a, positions_b, positions_c, &
    & nodes_b, ele_a, ele_b, local)
    real, dimension(:, :), intent(in) :: positions_a
    real, dimension(:, :), intent(in) :: positions_b
    real, dimension(:, :, :), intent(in) :: positions_c
    integer, dimension(:), intent(in) :: nodes_b
    integer, intent(in) :: ele_a
    integer, intent(in) :: ele_b
    logical, intent(in) :: local
\end{lstlisting}

\begin{description}[font=\ttfamily\bfseries,leftmargin=2.2\parindent,labelindent=1.7\parindent,noitemsep]
  \item[positions\_a] $d \times V_A$ array. Local mesh A node coordinates.
  \item[positions\_b] If \verb+local+ is true, a $d \times V_B$ array containing
    the local mesh B node coordinates. If \verb+local+ is false, a
    $d \times V_B^*$ array containing the received mesh B coordinates.
  \item[positions\_c] $d \times l_C \times E_C$. Intersection mesh C.
  \item[nodes\_b] \verb+positions_b(:, ele_b)+ are the mesh nodes associated
    with element B.
  \item[ele\_a] Index of element A in the local mesh A.
  \item[ele\_b] If \verb+local+ is true, index of element B in the local mesh B.
    If \verb+local+ is false, index of element B in the received mesh B.
  \item[local] If \verb+local+ is true then the calculation involves the local
    mesh B. If \verb+local+ is false then the calculation involves the received
    mesh B.
\end{description}

\section{Copyright}\label{sect:copyright}

\subsection{libsupermesh}

Except as noted below, libsupermesh is copyright (C) 2016 The University of
Edinburgh. The GNU Lesser General Public License version 2.1 can be found in
lgpl-2.1.txt.

\begin{lstlisting}[language=]
  This library is free software; you can redistribute it and/or
  modify it under the terms of the GNU Lesser General Public
  License as published by the Free Software Foundation; either
  version 2.1 of the License, or (at your option) any later version.

  This library is distributed in the hope that it will be useful,
  but WITHOUT ANY WARRANTY; without even the implied warranty of
  MERCHANTABILITY or FITNESS FOR A PARTICULAR PURPOSE.  See the GNU
  Lesser General Public License for more details.

  You should have received a copy of the GNU Lesser General Public
  License along with this library; if not, write to the Free Software
  Foundation, Inc., 51 Franklin Street, Fifth Floor, Boston, MA  02110-1301  USA
\end{lstlisting}

\subsection{Fluidity}

libsupermesh is derived from Fluidity git revision
4e6c1d2b022df3a519cdec120fad28e60d1b08d9 (dated 2015-02-25). libsupermesh
additionally uses code from Fluidity 4.1.11 -- see comments in src/Graphs.F90.
Fluidity copyright information (note that AUTHORS mentioned in the following has
been renamed to Fluidity\_AUTHORS):

\begin{lstlisting}[language=]
  Copyright (C) 2006 Imperial College London and others.
  
  Please see the AUTHORS file in the main source directory for a full list
  of copyright holders.

  Prof. C Pain
  Applied Modelling and Computation Group
  Department of Earth Science and Engineering
  Imperial College London

  amcgsoftware@imperial.ac.uk
  
  This library is free software; you can redistribute it and/or
  modify it under the terms of the GNU Lesser General Public
  License as published by the Free Software Foundation,
  version 2.1 of the License.

  This library is distributed in the hope that it will be useful,
  but WITHOUT ANY WARRANTY; without even the implied warranty of
  MERCHANTABILITY or FITNESS FOR A PARTICULAR PURPOSE.  See the GNU
  Lesser General Public License for more details.

  You should have received a copy of the GNU Lesser General Public
  License along with this library; if not, write to the Free Software
  Foundation, Inc., 59 Temple Place, Suite 330, Boston, MA  02111-1307
  USA
\end{lstlisting}

\subsection{TinyXML}

libsupermesh includes a modified version of TinyXML 2.6.2 (include/tinystr.h,
include/tinyxml.h, src/tinystr.cpp, src/tinyxml.cpp, src/tinyxmlerror.cpp, and
src/tinyxmlparser.cpp). Further information regarding TinyXML can be found in
TinyXML\_readme.txt. TinyXML 2.6.2 license information:

\begin{lstlisting}[language=]
www.sourceforge.net/projects/tinyxml
Original code by Lee Thomason (www.grinninglizard.com)

This software is provided 'as-is', without any express or implied
warranty. In no event will the authors be held liable for any
damages arising from the use of this software.

Permission is granted to anyone to use this software for any
purpose, including commercial applications, and to alter it and
redistribute it freely, subject to the following restrictions:

1. The origin of this software must not be misrepresented; you must
not claim that you wrote the original software. If you use this
software in a product, an acknowledgment in the product documentation
would be appreciated but is not required.

2. Altered source versions must be plainly marked as such, and
must not be misrepresented as being the original software.

3. This notice may not be removed or altered from any source
distribution.
\end{lstlisting}

\subsection{libspatialindex}

libsupermesh includes a modified version of libspatialindex 1.8.5
(spatialindex-1.8.5/). libsupermesh additionally uses modified versions of code
from libspatialindex -- see comments in
include/R-Tree\_Intersection\_Finder\_C++.h and
src/R-Tree\_Intersection\_Finder\_C++.cpp. Further information regarding
libspatialindex can be found in spatialindex-1.8.5/. libspatialindex 1.8.5
license information:

\begin{lstlisting}[language=]
License (MIT)
------------------------------------------------------------------------------

::

    Permission is hereby granted, free of charge, to any person obtaining a
    copy of this software and associated documentation files (the "Software"),
    to deal in the Software without restriction, including without limitation
    the rights to use, copy, modify, merge, publish, distribute, sublicense,
    and/or sell copies of the Software, and to permit persons to whom the
    Software is furnished to do so, subject to the following conditions:

    The above copyright notice and this permission notice shall be included in
    all copies or substantial portions of the Software.

    THE SOFTWARE IS PROVIDED "AS IS", WITHOUT WARRANTY OF ANY KIND, EXPRESS OR
    IMPLIED, INCLUDING BUT NOT LIMITED TO THE WARRANTIES OF MERCHANTABILITY,
    FITNESS FOR A PARTICULAR PURPOSE AND NONINFRINGEMENT. IN NO EVENT SHALL THE
    AUTHORS OR COPYRIGHT HOLDERS BE LIABLE FOR ANY CLAIM, DAMAGES OR OTHER
    LIABILITY, WHETHER IN AN ACTION OF CONTRACT, TORT OR OTHERWISE, ARISING
    FROM, OUT OF OR IN CONNECTION WITH THE SOFTWARE OR THE USE OR OTHER
    DEALINGS IN THE SOFTWARE.
\end{lstlisting}

\subsection{Rtree}

libsupermesh uses modified versions of code from Rtree 0.4.3. See comments
in \linebreak include/R-Tree\_Intersection\_Finder\_C++.h and
src/R-Tree\_Intersection\_Finder\_C++.cpp. See also \linebreak
Rtree\_CREDITS.txt. Rtree 0.4.3 copyright information from files
rtree/gispyspatialindex.cc and rtree/wrapper.cc:

\begin{lstlisting}[language=]
  =============================================================================
  Rtree spatial index. Copyright (C) 2007 Sean C. Gillies
 
  This library is free software; you can redistribute it and/or modify it under
  the terms of the GNU Lesser General Public License as published by the Free
  Software Foundation; either version 2.1 of the License, or (at your option)
  any later version.
 
  This library is distributed in the hope that it will be useful, but WITHOUT
  ANY WARRANTY; without even the implied warranty of MERCHANTABILITY or FITNESS
  FOR A PARTICULAR PURPOSE.  See the GNU Lesser General Public License for more
  details.
 
  You should have received a copy of the GNU Lesser General Public License 
  along with this library; if not, write to the Free Software Foundation, Inc.,
  51 Franklin Street, Fifth Floor, Boston, MA  02110-1301  USA
 
  Contact email: sgillies@frii.com
  =============================================================================
\end{lstlisting}

\subsection{UseLATEX}

libsupermesh includes UseLATEX. See also cmake/UseLATEX.cmake. CMakeUserUseLATEX
copyright information:

\begin{lstlisting}[language=] 
  Copyright 2004, 2015 Sandia Corporation.
  Under the terms of Contract DE-AC04-94AL85000, there is a non-exclusive
  license for use of this work by or on behalf of the U.S. Government.
 
  This software is released under the BSD 3-Clause License.
 
  Redistribution and use in source and binary forms, with or without
  modification, are permitted provided that the following conditions are
  met:
 
  1. Redistributions of source code must retain the above copyright notice,
  this list of conditions and the following disclaimer.
 
  2. Redistributions in binary form must reproduce the above copyright
  notice, this list of conditions and the following disclaimer in the
  documentation and/or other materials provided with the distribution.
 
  3. Neither the name of the copyright holder nor the names of its
  contributors may be used to endorse or promote products derived from this
  software without specific prior written permission.
 
  THIS SOFTWARE IS PROVIDED BY THE COPYRIGHT HOLDERS AND CONTRIBUTORS "AS
  IS" AND ANY EXPRESS OR IMPLIED WARRANTIES, INCLUDING, BUT NOT LIMITED TO,
  THE IMPLIED WARRANTIES OF MERCHANTABILITY AND FITNESS FOR A PARTICULAR
  PURPOSE ARE DISCLAIMED. IN NO EVENT SHALL THE COPYRIGHT HOLDER OR
  CONTRIBUTORS BE LIABLE FOR ANY DIRECT, INDIRECT, INCIDENTAL, SPECIAL,
  EXEMPLARY, OR CONSEQUENTIAL DAMAGES (INCLUDING, BUT NOT LIMITED TO,
  PROCUREMENT OF SUBSTITUTE GOODS OR SERVICES; LOSS OF USE, DATA, OR
  PROFITS; OR BUSINESS INTERRUPTION) HOWEVER CAUSED AND ON ANY THEORY OF
  LIABILITY, WHETHER IN CONTRACT, STRICT LIABILITY, OR TORT (INCLUDING
  NEGLIGENCE OR OTHERWISE) ARISING IN ANY WAY OUT OF THE USE OF THIS
  SOFTWARE, EVEN IF ADVISED OF THE POSSIBILITY OF SUCH DAMAGE.
\end{lstlisting}

\section{Acknowledgements}

Development of libsupermesh was funded by ARCHER eCSE03-8, ``Parallel
supermeshing for multimesh modelling''.

libsupermesh is derived from Fluidity git revision
4e6c1d2b022df3a519cdec120fad28e60d1b08d9, and additionally uses code from
TinyXML 2.6.2, libspatialindex 1.8.5, Rtree 0.4.3, and UseLATEX (see section
\ref{sect:copyright}).

Gmsh \citep{geuzaine2009} was used to generate a number of the test meshes
contained in src/tests/data/. The Fluidity fldecomp tool, which uses the METIS
graph partitioner\footnote{\url{http://glaros.dtc.umn.edu/gkhome/views/metis}},
was used to decompose a number of the test meshes contained in src/tests/data/.

\bibliography{doc/bibliography}
\bibliographystyle{plainnat}

\end{document}
